%% start of file `template.tex'.
%% Copyright 2006-2015 Xavier Danaux (xdanaux@gmail.com), 2020-2022 moderncv maintainers (github.com/moderncv).
%
% This work may be distributed and/or modified under the
% conditions of the LaTeX Project Public License version 1.3c,
% available at http://www.latex-project.org/lppl/.


\documentclass[10pt,a4paper,roman]{moderncv}        % possible options include font size ('10pt', '11pt' and '12pt'), paper size ('a4paper', 'letterpaper', 'a5paper', 'legalpaper', 'executivepaper' and 'landscape') and font family ('sans' and 'roman')

% moderncv themes
\moderncvstyle{classic}                             % style options are 'casual' (default), 'classic', 'banking', 'oldstyle' and 'fancy'
\moderncvcolor{burgundy}                               % color options 'black', 'blue' (default), 'burgundy', 'green', 'grey', 'orange', 'purple' and 'red'
%\renewcommand{\familydefault}{\sfdefault}         % to set the default font; use '\sfdefault' for the default sans serif font, '\rmdefault' for the default roman one, or any tex font name
%\nopagenumbers{}                                  % uncomment to suppress automatic page numbering for CVs longer than one page

% adjust the page margins
\usepackage[scale=0.75]{geometry}
\setlength{\footskip}{136.00005pt}                 % depending on the amount of information in the footer, you need to change this value. comment this line out and set it to the size given in the warning
%\setlength{\hintscolumnwidth}{3cm}                % if you want to change the width of the column with the dates
%\setlength{\makecvheadnamewidth}{10cm}            % for the 'classic' style, if you want to force the width allocated to your name and avoid line breaks. be careful though, the length is normally calculated to avoid any overlap with your personal info; use this at your own typographical risks...

% font loading
% for luatex and xetex, do not use inputenc and fontenc
% see https://tex.stackexchange.com/a/496643
\ifxetexorluatex
  \usepackage{fontspec}
  \usepackage{unicode-math}
  \defaultfontfeatures{Ligatures=TeX}
  \setmainfont{Latin Modern Roman}
  \setsansfont{Latin Modern Sans}
  \setmonofont{Latin Modern Mono}
  \setmathfont{Latin Modern Math} 
\else
  \usepackage[utf8]{inputenc}
  \usepackage[T1]{fontenc}
  \usepackage{lmodern}
\fi

% document language
\usepackage[english]{babel}  % FIXME: using spanish breaks moderncv

% personal data
\name{MALIK}{ARHAM}
\title{Étudiant/ Informatique}                               % optional, remove / comment the line if not wanted
\born{8 Février 2004}                                 % optional, remove / comment the line if not wanted
\address{66 Allée des gémeaux}{93600 Aulnay-sous-bois}{France}% optional, remove / comment the line if not wanted; the "postcode city" and "country" arguments can be omitted or provided empty
\phone[mobile]{+33 ~06~18~15~98~50}                 % optional, remove / comment the line if not wanted; the optional "type" of the phone can be "mobile" (default), "fixed" or "fax"
\email{arhammalik225@gmail.com}                               % optional, remove / comment the line if not wanted


% Social icons
\social[github]{hisoka204}                              % optional, remove / comment the line if not wanted
\social[whatsapp]{0618159850}                     % optional, remove / comment the line if not wanted



%   to show numerical labels in the bibliography (default is to show no labels)
%\makeatletter\renewcommand*{\bibliographyitemlabel}{\@biblabel{\arabic{enumiv}}}\makeatother
\renewcommand*{\bibliographyitemlabel}{[\arabic{enumiv}]}
%   to redefine the bibliography heading string ("Publications")
%\renewcommand{\refname}{Articles}

% bibliography with mutiple entries
%\usepackage{multibib}
%\newcites{book,misc}{{Books},{Others}}
%----------------------------------------------------------------------------------
%            content
%----------------------------------------------------------------------------------
\begin{document}
%\begin{CJK*}{UTF8}{gbsn}                          % to typeset your resume in Chinese using CJK
%-----       resume       ---------------------------------------------------------
\makecvtitle

\section{Formation}
\cventry{2022 - now }{Licence informatique,}{Université Paris 8}{}{\textit{}}{Licence Informatique avec Mineur Math}  % arguments 3 to 6 can be left empty
\cventry{2020 - 2022}{bac général,}{lycée Jean Zay}{Aulnay-Sous-Bois}{\textit{}}{Spécialité Physique-Chimie/mathématiques}

\section{Experience}
\cventry{Mai 2023 - Juillet 2023 }{Vendeur}{polyvalent}{A.R International}{Pantin France}{
\begin{itemize}
\item Compétences en Communication,
\item Travail d'Équipe,
\item Adatation a toute situation,
\end{itemize}}

\cventry{Juin 2021 - Aout 2021 }{Équipier polyvalent}{Mcdonalds}{Gonesse}{}{
\begin{itemize}
\item Gestion du Stress,
\item Capacité à Apprendre,
\item Service à la Clientèle,
\end{itemize}}

\cventry{Mars 2019}{Stage entrprise}{informatique}{Paris}{France}{
\begin{itemize}
\item Vie en entreprise,
\item différents role au sein de l’entreprise
\end{itemize}}

\section{Langues}
\cvitemwithcomment{Anglais}{médium}{anglais technique}
\cvitemwithcomment{Français}{natif}{langue maternel}
\cvitemwithcomment{urdu}{natif}{langue maternel}
\cvitemwithcomment{Allemand }{faible}{quelque notion}

\section{Compétences en informatique}
\cvitem{OS}{Debian, powershell}
\cvitem{Programmation }{HTML, CSS, SQL, Bash, C, python, racket, Ocaml}

\section{Centre d'intérêt}
\cvitem{La Mode}{J'aime beaucoup ce qui ce passe dans le monde de la mode et tout particulièrement dans les sneakers.}
\cvitem{Le Basket}{Je suis un grand Fan de Stephen Curry.}
\cvitem{Technologie}{Je suis un étudiant en Informatique donc j'adore programmer et jouer à des jeux video.}
\cvitem{Cinema}{Je suis un grand fan de Tarantino et Christopher Nolan (et les film d'animation GHIBLI). }



\end{document}


%% end of file `template.tex'